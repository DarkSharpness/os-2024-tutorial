
\chapter{总述}
\section{作业内容}
本系列作业旨在引导大家通过修改Linux内核的实现感受操作系统的功能。
作业具体包括以下6个大类:启动、系统调用与特权态执行、内存管理、文件系统、网络与外部设备、安全。
每个大类进一步根据难度和内容分为基础、实践、设计、综合。前三部分的内容如表\ref{fig:intro:all}所示。
综合部分的选题每一个选题横跨两个子项目,内容和目标如表\ref{fig:intro:mixed}所示。

\begin{table}[htbp]
\centering

\caption{分项内容}
\label{fig:intro:all}
\begin{tabular}{c|c|c|c}

\hline
分项                                                 & 基础      & 实践  & 设计         \\ \hline
系统启动         & Read ACPI Table     & Hack ACPI Table & UEFI运行时服务 \\ \hline
\begin{tabular}[c]{@{}c@{}}系统调用\\ 特权态\end{tabular} & 用户定义的系统调用     & vDSO & 无需中断的系统调用     \\ \hline
内存管理                                               & 页表和文件页      & 内存文件系统 & 内存压力导向的内存管理        \\ \hline
文件系统                                               & inode和扩展属性管理     & FUSE & 用户空间下的内存磁盘       \\ \hline
\begin{tabular}[c]{@{}c@{}}网络\\ 外部设备\end{tabular}  & tcpdump和socket管理  & NCCL & DPDK       \\ \hline
% \begin{tabular}[c]{@{}c@{}}安全\\ 虚拟化\end{tabular}  & 常见的溢出攻击和侧信道 & 虚拟机调用Hypercall & 虚拟机内存访问加速        \\ \hline
\end{tabular}
\end{table}


\begin{table}[htbp]
\centering

\caption{综合部分分项内容}
\label{fig:intro:mixed}
\begin{tabular}{c|c}
\hline
项目              & 涉及分项         \\ \hline
使用RDMA的远程内存Swap & 内存管理、网络与外部设备 \\ \hline
用户程序嵌入内核执行 & 系统调用、内存管理 \\ \hline
内核空间下的内存磁盘 & 内存管理、文件系统 \\ \hline
机密虚拟机Trapless的宿主协同内存管理 & 内存管理、安全与虚拟化 \\ \hline

\end{tabular}

\end{table}

\section{作业安排(初步)}

在本学期的作业中,我们将会定期安排大家进行习题课完成答疑和基本环境配置。小作业的初步计分规则如下(满分100分),正式发布在3月1日。

\begin{enumerate}
    \item 基础部分:每个分项5分,共30分,必选;
    \item 实践部分:每个分项10分,共60分;
    \item 设计部分:每个分项15分,共90分;
    \item 综合部分:每个分项50分,选一个共50分。
\end{enumerate}

作业层级之间存在关联,即实践部分的内容可能会依赖基础部分的内容,设计部分的内容可能会依赖实践部分的内容,综合部分的内容可能会依赖设计部分的内容。因此,建议大家按照顺序完成作业。
分数的设定同样存在层级关联,实践部分需要基础部分完全完成才可以计算,设计部分需要至少完成2个实践部分才可以计算,综合部分需要至少完成1个设计部分才可以计算。
综合部分可能需要用到特别的硬件,如果需要使用特殊的硬件请联系助教。

\paragraph*{诚信原则}
如果出现以下情况,将会扣分:
\begin{enumerate}
    \item 某一项作业与他人完全一致(即使是两人都是用大模型生成):计小作业0分。
    \item 某一项作业疑似使用大模型生成且没有对代码的合理解释:计该项作业0分。
\end{enumerate}

\paragraph*{作业提交}
为了方便大家更加使用符合自己使用习惯的内核,我们允许使用任何Linux 5.x.x的内核版本完成作业。(不建议使用6.0以及更高版本,也不建议使用5.0之前的4.x版本)。请在4月15日前提交你的基础版本号。
对于每一个作业,你需要提交按照作业要求上载明的内容(部分的作业子项要求报告)和对应与base版本的diff文件。提交的文件格式为pdf和patch文件。
所有作业的截止时间均为本学期16周周日晚上23:59,但是自第8周开始,你可以提前提交作业(Code Review仍然安排在16周),提前提交作业不会有额外的加分。

\paragraph*{Code Review}
Code Review将会在16周进行,具体时间和地点将会在之后公布。Code Review的内容将会包括对你的代码的评价和对你的报告的评价。Code Review的分数将会占到你分项的30\%。
\textbf{你不应当认为Code Review是容易获得满分的。}

\paragraph*{额外加分}
我们鼓励大家通过向本Repo提交PR来提高你的分数。每一个PR将会有0.1至2分不等的加分(加小作业部分分),分数取决于表述的质量和作用。
PR的内容可以是对本Repo的文档的修正,对作业的补充,对作业的环境配置提供的建议等等。
修正错别字、表述错误至多可以提交10次,每次0.1分。
对作业描述的补充、环境配置按照书写的质量给分。
其余类型的PR提交后请联系助教确认加分。
要求释疑很大概率不会获得加分。
重复提交不给分(重复修复同一个错别字、重复给出相似的作业描述补充等),按照PR的编号决定加分对象。加分给编号较小的人。
最终加分不超过20分。