\chapter*{实践任务的背景}

在21世纪20年代,机器学习的极具发展、大模型的快速迭代使得操作系统相关的内容成为“不受大家待见”的东西。
学术界急速转向机器学习系统相关系统使得大家对于传统硬核的操作系统失去了兴趣。
不仅班级内发生了转向,大多数实验室也在这一期间放弃原有传统系统研究而转向新兴与大模型结合的系统研究。
为了适应这一潮流,对于对计算机系统没有兴趣的同学,助教们经过讨论认为需要将重点放在“如何使用好操作系统”和“如何理解操作系统的能力”两个部分。
为此,助教们结合2019、2020、2021、2022级日常作业的内容,将小作业实践总结的目标进行整合,整理出了这一版实践的目标手册。

由于时间仓促,这一版本内容相对比较单薄。期待同学们在作业发展的过程中不断迭代增加内容。

第一版实践手册牵头讨论和实践的同学是郑文鑫(2017)、王鲲鹏(2022)、房诗涵(2022)、徐子绎(2022)、潘屹(2022)。