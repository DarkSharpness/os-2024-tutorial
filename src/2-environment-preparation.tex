\chapter{项目环境准备}

\section{基准代码}
本系列的基准代码规则如下:
\begin{enumerate}
    \item \textbf{Linux}: 任何体系结构下高于5.10.20版本的Linux内核都是可行的选项,你可以从\url{https://www.kernel.org/}中选择合适的版本。对于\texttt{initramfs},你可以选择任何一个发行版。如果你选择直接使用qcow或者其他完全预制好的Ubuntu镜像,你需要手动在内部更换内核以确保你的代码能够生效。
    \item \textbf{QEMU}: 版本>=7。
    \item \textbf{EDK2}: 跟随\url{https://github.com/tianocore/edk2}选择合适的版本。其中Ubuntu\_GCC5并不意味着GCC版本必须是5。
\end{enumerate}

\section{在QEMU中启动Linux镜像}
\begin{remark}
简单起见你可以直接使用预制作好的\texttt{qcow} Ubuntu镜像作为磁盘镜像由qemu直接引导,具体教程:\url{https://documentation.ubuntu.com/public-images/en/latest/public-images-how-to/launch-qcow-with-qemu/}。如果你选用这个方案,你只需要确保QEMU正常安装即可。
\end{remark}

\begin{remark}
    本区域中.x.x部分请填入自己对应的版本号。
\end{remark}



\subsection{QEMU安装(二选一)}
\subsubsection{软件包安装}
\begin{lstlisting}[language=bash]
sudo apt-get install qemu-system-x86
qemu-system-x86_64 --version
\end{lstlisting}

\subsubsection{源码编译安装}
\begin{lstlisting}[language=bash]
# 安装依赖
sudo apt-get install git libglib2.0-dev libfdt-dev libpixman-1-dev zlib1g-dev ninja-build

# 下载编译
mkdir ~/kvm && cd ~/kvm
wget https://download.qemu.org/qemu-7.2.0.tar.xz
tar -xf qemu-7.2.0.tar.xz
cd qemu-7.2.0
./configure --target-list=x86_64-softmmu
make -j$(nproc)
\end{lstlisting}

\subsubsection{常见依赖问题}
\begin{itemize}
    \item \textbf{ERROR: Cannot find Ninja} → 执行 \texttt{sudo apt install ninja-build}
    \item \textbf{ERROR: glib-2.56 required} → 执行 \texttt{sudo apt install libglib2.0-dev}
    \item \textbf{Dependency "pixman-1" not found} → 执行 \texttt{sudo apt install libpixman-1-dev}
\end{itemize}

\subsection{Linux内核编译}
\begin{lstlisting}[language=bash]
# 安装依赖
sudo apt-get install git fakeroot build-essential ncurses-dev xz-utils libssl-dev bc flex libelf-dev bison 
mkdir ~/kernel && cd ~/kernel
wget https://git.kernel.org/pub/scm/linux/kernel/git/stable/linux.git/snapshot/linux-5.x.x.tar.gz
tar -xf linux-5.x.x.tar.gz
cd linux-5.x.x

make defconfig    # 使用默认配置
make -j$(nproc)   # 开始编译
\end{lstlisting}

\textbf{编译输出文件}: \texttt{arch/x86/boot/bzImage}

在编译过程中,可能会发生编译错误的情况。这可能是 GCC 版本过高导致的。这里给出一种可行的组合: Linux-5.19.17, 使用 GCC-12 进行编译,操作如下:

\begin{lstlisting}[language=bash]
sudo apt install gcc-12 g++-12 # 安装gcc-12
sudo update-alternatives --install /usr/bin/gcc gcc /usr/bin/gcc-12 70
sudo update-alternatives --install /usr/bin/g++ g++ /usr/bin/g++-12 70 # 切换 gcc 默认版本
\end{lstlisting}
经过这些操作后,应该可以正常编译 Linux 内核了。

\subsection{BusyBox编译}
这里可以采用其他发行版(例如Ubuntu发行版等)。
\begin{lstlisting}[language=bash]
mkdir ~/kvm && cd ~/kvm
wget https://busybox.net/downloads/busybox-1.x.x.tar.bz2
tar -xf busybox-1.x.x.tar.bz2
cd busybox-1.x.x

# 配置命令
make menuconfig
# 选中: Settings > Build Options > [*] Build static binary
# 取消: Shells > [ ] Job control

make -j$(nproc) && make install
\end{lstlisting}

\subsection{EDK2编译}
EDK2的可编译源码在git submodule里,因此相对简单可靠的方式就是直接按照官网所说的那样:
\begin{lstlisting}[language=bash]
git clone -b stable/202408 https://github.com/tianocore/edk2.git
cd edk2
git submodule update --init
\end{lstlisting}
在编译之前,需要在系统里安装一些基础的包(例如base-devel uuid、nasm、acpica),然后就可以编译BaseTools。
\begin{lstlisting}[language=bash]
source edksetup.sh
make -C BaseTools -j$(nproc)
\end{lstlisting}
然后编辑\texttt{Conf/target.txt}(参考\url{https://github.com/tianocore/tianocore.github.io/wiki/How-to-build-OVMF}):
\begin{lstlisting}
ACTIVE_PLATFORM       = OvmfPkg/OvmfPkgIa32X64.dsc
TARGET_ARCH           = IA32 X64
TOOL_CHAIN_TAG        = GCC5
\end{lstlisting}
最后,执行:
\begin{lstlisting}[language=bash]
build
\end{lstlisting}
然后就可以在\texttt{Build/Ovmf3264/DEBUG\_GCC5/FV/}目录找到\texttt{OVMF\_CODE.fd}和\texttt{OVMF\_VARS.fd}两个编译出的OVMF文件。(这里的Ovmf3264和DEBUG\_GCC5和编译选项有关,也是在\texttt{Conf/target.txt}里面配置。)

\subsection{制作initramfs}
\subsubsection{创建启动脚本}
\begin{lstlisting}[language=bash]
cd busybox-1.x.x/_install/
mkdir -p proc sys dev tmp
echo -e '#!/bin/sh\nmount -t proc none /proc\nmount -t sysfs none /sys\nmount -t tmpfs none /tmp\nmount -t devtmpfs none /dev\necho "Hello Linux!"\nexec /bin/sh' > init
chmod +x init
\end{lstlisting}

\subsubsection{打包文件系统}
\begin{lstlisting}[language=bash]
find . -print0 | cpio --null -ov --format=newc | gzip -9 > ../initramfs.cpio.gz
\end{lstlisting}

\subsection{启动QEMU}
\subsubsection{图形界面启动}
\begin{lstlisting}[language=bash]
qemu-system-x86_64 \
    -kernel ./linux-5.x.x/arch/x86/boot/bzImage \
    -initrd ./busybox-1.x.x/initramfs.cpio.gz \
    -append "init=/init"
\end{lstlisting}

\subsubsection{字符界面启动}
\begin{lstlisting}[language=bash]
qemu-system-x86_64 \
    -kernel ./linux-5.19/arch/x86/boot/bzImage \
    -initrd ./busybox-1.35.0/initramfs.cpio.gz \
    -nographic \
    -append "init=/init console=ttyS0"
\end{lstlisting}
如果使用EDK2的话,可以在参数里加上两行:
\begin{lstlisting}[language=bash]
    -drive if=pflash,format=raw,unit=0,file="${OVMF_CODE}",readonly=on \
    -drive if=pflash,format=raw,unit=1,file="${PLAYGROUND_DIR}/OVMF_VARS.fd" \
\end{lstlisting}

\subsection{验证}
\begin{itemize}
    \item 检查内核版本: \texttt{uname -a}
    \item 查看启动参数: \texttt{cat /proc/cmdline}
    \item 测试基本命令: \texttt{ls}, \texttt{mount}
\end{itemize}

\section{Linux环境部署 EDK2 开发环境}

python的版本要求>=3.11,

\subsubsection{下载源码}
\begin{lstlisting}[language=bash]
git clone https://github.com/tianocore/edk2.git
cd edk2
git submodule update --init  #大概需要几分钟
cd ..
\end{lstlisting}

\subsubsection{网上找到的可能需要安装的一堆东西}
\begin{lstlisting}[language=bash]
sudo apt-get install build-essential uuid-dev
sudo apt-get install uuid-dev nasm bison flex
sudo apt-get install libx11-dev x11proto-xext-dev libxext-dev
\end{lstlisting}

将 NASM 更新到 2.15.x 以上版本,当前的edk2项目需要依赖nasm2.15以上版本,否则编译会报错
nasm各版本下载链接:http://www.nasm.us/pub/nasm/releasebuilds/?C=M;O=D

\begin{lstlisting}[language=bash]
cd nasm-2.16
./autogen.sh
./configure
make
make install
nasm --version
\end{lstlisting}

\subsubsection{make一下BaseTools}

然后cd到edk2/BaseTools下
\begin{lstlisting}[language=bash]
make
\end{lstlisting}

显示ok则成功。
注意这里所有的文件应该是LF,不能是CRLF,不然会出问题。可以先检查一下。

然后到edk2的conf目录下,创建target.txt(template在readme里面)
然后修改:

TARGET\_ARCH           = X64
TOOL\_CHAIN\_TAG        = GCC5

\subsubsection{编译UEFI模拟器和UEFI工程}

先到edk2目录下,然后执行

\begin{lstlisting}[language=bash]
sourse edksetup.sh
build
\end{lstlisting}

然后到target.txt里面改一下:

ACTIVE\_PLATFORM = OvmfPkg/OvmfPkgX64.dsc

\subsubsection{运行}

首先下载一个ubuntu的iso镜像,比如ubuntu-24.04.1-live-server-amd64.iso
然后找到自己的OVMF.fd,一般在edk2/Build/OvmfX64/DEBUG\_GCC5/FV/OVMF.fd下
下面的bios就是跑一下自己编译出的UEFI模拟器,cdrom就是iso镜像,m 4096是内存大小,enable-kvm是开启虚拟化加速
正常来说sudo可以不加,m可以不加,kvm的指令不需要加,但是我不加的话会报错,所以加上了

\begin{lstlisting}[language=bash]
sudo qemu-system-x86_64 -m 4096 -cdrom ~/ubuntu-24.04.1-live-server-amd64.iso -bios ~/OVMF.fd -enable-kvm
\end{lstlisting}
