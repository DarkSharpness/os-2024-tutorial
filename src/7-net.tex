\chapter{网络与外部设备}
\section{tcpdump和socket管理}
Linux内核中提供了抓包等服务作为内核的网络套件的一部分。

目标:实现如下内容
\begin{enumerate}
    \item 按照一定条件进行抓包(要求修改tcpdump,不允许直接用现成的参数)
    \item 内核管理socket的安全性实现per-thread的socket fairness管理。
\end{enumerate}

测试:多Socket读写的管理(测试p99延迟)

\section{高速通信库:NCCL}
现代的高性能计算已经不能由单节点计算满足,需要多个节点多台机器进行并行计算。随着单节点算力的不断上升,并行计算时的通信成为了影响整个计算集群效率的关键因素。
传统的网络通信已经不能满足这一通信需求,各个计算卡的厂商普遍提供了自己的高速互联协议(如英伟达的NCCL、华为的HCCL)。

目标:通过NCCL实现简单的信息传输,比较和传统使用以太网的速度、正确性差异。



\section{数据平面开发套件:DPDK}
当前的网络协议都在内核空间中,因此涉及网络的操作都需要通过特权切换进入内核。
对于需要高速通信的程序,反复的上下文切换显然不符合要求。
不过,由于完整实现基于DPDK的通信所需要的代码量太大,我们在这里进行简化,只需要实现简单的网络即可。

目标:实现用户态基于DPDK的网络QoS管理,实现多个进程访问网络的带宽平衡。